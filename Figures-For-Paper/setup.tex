%% For various important things
\usepackage{amssymb,amsfonts,amsmath}

%% For mathscr
\usepackage[mathscr]{eucal}

%% For boldsymbol
\usepackage{amsbsy}

%% For \llbracket and \rrbracket, \varoast, \varoslash
\usepackage{stmaryrd}

%% For bm (bold math)
\usepackage{bm}

% For \set
\usepackage{braket}  

%% For special document commands, i.e., macros
\usepackage{xparse}

%% Intelligent spacing for text macros
\usepackage{xspace}

%% For color
\usepackage{xcolor}

%% For hyperlinks
\usepackage{hyperref}

%% Subfloats
\usepackage[font=footnotesize,justification=Centering,singlelinecheck=false]{subfig}

%% For smart cross-references
\usepackage{cleveref}
\crefformat{footnote}{#2\footnotemark[#1]#3}

%% Setup for tikz plot
\usepackage{pgfplots}
\usepackage{pgfplotstable}
\usetikzlibrary{plotmarks}
\usepgfplotslibrary{colorbrewer}
\usepgfplotslibrary{statistics}
\usepgfplotslibrary{fillbetween}
\pgfplotsset{compat=1.15}


\pgfplotsset{cycle list/Set1-8} % Needed to initialize colormap before indexing later!
\pgfplotstableset{col sep=comma}

%% Algorithms
\usepackage{algorithm, algpseudocode}
\Crefname{ALC@unique}{Line}{Lines}
\algnewcommand{\LineComment}[1]{\State \(\triangleright\) #1}

%% Comments
\usepackage{fixme}
\fxsetup{
  status=draft,
  nomargin,
  inline,
  theme=color,
}
\fxsetup{marginface=\linespread{1}\footnotesize}
\FXRegisterAuthor{tk}{tke}{TK}
\FXRegisterAuthor{bl}{ble}{BL}

%% ----------------------------------------------------------------------
%% Macros
%% ----------------------------------------------------------------------

% Single-line "for" statement in algorithm
\NewDocumentCommand{\LineFor}{m m}{%
  \State\textbf{for} {#1}, \textbf{do} {#2}, \textbf{end}
  }

% Quad-text-quad
\NewDocumentCommand \qtext {m} {\quad\text{#1}\quad}

% Reals
\NewDocumentCommand \Real {} {\mathbb{R}}

% Naturals
\NewDocumentCommand \Natural {} {\mathbb{N}}

% Tensor
\NewDocumentCommand \T { O{} m } {\boldsymbol{#1\mathscr{\MakeUppercase{#2}}}}

% Transpose checker for macro
\NewDocumentCommand \TCheck { m } {\IfBooleanT{#1}{^{\intercal}}}

% Matrix
\NewDocumentCommand \Mx { O{} m G{\BooleanFalse} t'} {{\bm{#1\mathbf{\MakeUppercase{#2}}}%
    \IfBooleanT{#3}{^{\intercal}}%
    \IfBooleanT{#4}{^{\intercal}}%
  }} 

% Vector
\NewDocumentCommand \V { O{} m } {{\bm{#1\mathbf{\MakeLowercase{#2}}}}} 

% --- LS Subproblem: ---
\NewDocumentCommand \Z { t~ } {\Mx[\IfBooleanT{#1}{\tilde}]{Z}}
\NewDocumentCommand \B { t' t~ t*} {\Mx[\IfBooleanT{#2}{\tilde}]{B}\IfBooleanT{#1}{^{\intercal}}\IfBooleanT{#3}{_{\ast}}}
\NewDocumentCommand \A { } {\Mx{A}}
\NewDocumentCommand \Xmat { t' t~} {\Mx[\IfBooleanT{#2}{\tilde}]{X}{#1}}
\NewDocumentCommand \Om { t! } {\Mx[\IfBooleanT{#1}{\bar}]{\Omega}}
\NewDocumentCommand \bvec { t~ t* } {\V[\IfBooleanT{#1}{\tilde}]{\alpha}\IfBooleanT{#2}{_{\ast}}}
\NewDocumentCommand \xvec { } {\V{\nu}}
\NewDocumentCommand \Vp { } {\V{p}}
\NewDocumentCommand \Multi { } {\text{\sc multinomial}}
\NewDocumentCommand \RandSample { } {\text{\sc randsample}}
\NewDocumentCommand \RandSampleCombo { } {\text{\sc randsamplecombo}}
\NewDocumentCommand \Prob { } {\text{\rm Pr}}
\NewDocumentCommand \Exp { } {\mathbb{E}}
\DeclareMathOperator{\range}{range}
\NewDocumentCommand \levZ { s } {\IfBooleanT{#1}{\bar}\ell_i(\Z)}
\NewDocumentCommand \levAk { } {\ell_{i_k}(\Ak)}

\NewDocumentCommand \Vl { } {\V{\ell}}
\NewDocumentCommand \Vw { } {\V{w}}
\NewDocumentCommand \Vmax { } {\V{\alpha}}

\DeclareDocumentCommand \det { } {\text{\rm\sffamily det}}
\NewDocumentCommand \rnd { } {\text{\rm\sffamily rnd}}

\NewDocumentCommand \ndet { } {s_{\text{\rm\sffamily det}}} 
\NewDocumentCommand \sdet { } {s_{\text{\rm\sffamily det}}} 
\NewDocumentCommand \srnd {s } {\IfBooleanT{#1}{\bar}s_{\text{\rm\sffamily rnd}}} 
\NewDocumentCommand \pdet { } {p_{\text{\rm\sffamily det}}} 

%\NewDocumentCommand{\pk}{G{k}}{\V{p}^{(#1)}}
%\NewDocumentCommand{\pke}{G{k} G{i_k}}{p^{(#1)}_{#2}}
\NewDocumentCommand{\pk}{G{k}}{\V{p}_{#1}}
\NewDocumentCommand{\pke}{G{k} G{i_k}}{(\pk{#1})_{#2}}
%\NewDocumentCommand{\pke}{G{k} G{i_k}}{p_{#1,#2}}

\NewDocumentCommand \DetSet { } {\mathcal{D}} %{\Omega_{\text{\sffamily det}}} 
\NewDocumentCommand \DetSetk { G{k} } {\mathcal{\bar D}_{#1}} 
\NewDocumentCommand \DetMat { } {\Mx{\Omega}_{\text{\rm\sffamily det}}} 

\NewDocumentCommand{\resid}{}{\mathcal{R}} 
\NewDocumentCommand{\rdet}{}{\mathcal{R}_{\text{\rm\sffamily det}}}
\NewDocumentCommand{\rrnd}{}{\mathcal{R}_{\text{\rm\sffamily rnd}}}


% Alg Names
\NewDocumentCommand{\SKRPLEV}{}{\textsc{SkrpLev}}
\NewDocumentCommand{\DIDX}{}{\textsc{DetSkrp}}
\NewDocumentCommand{\SIDX}{}{\textsc{RndSkrp}}
\NewDocumentCommand{\CIDX}{}{\textsc{CombineRepeats}}

\NewDocumentCommand{\idxdet}{}{\texttt{idet}}
\NewDocumentCommand{\idxrnd}{}{\texttt{irnd}}
\NewDocumentCommand{\wgtrnd}{}{\texttt{wrnd}}
\NewDocumentCommand{\idx}{}{\texttt{idx}}
\NewDocumentCommand{\wgt}{}{\texttt{wgt}}



\NewDocumentCommand{\cpals}{}{{CP-ALS}\xspace}
\NewDocumentCommand{\cprand}{}{{CP-ARLS}\xspace}
\NewDocumentCommand{\cprandlev}{}{{CP-ARLS-LEV}\xspace}

% --- Leverage Score Notation: ---

% --- Data Tensor: X ---

% Tensor X
\NewDocumentCommand \X { } {\T{X}}

% Mode-k Unfolding of tensor X
\NewDocumentCommand \Xk { G{k} } {\Mx{X}_{(#1)}}

% bwlarsen: This conflicts with the greek letter \xi
%    used for the number of non-zeros
% Single element of tensor X
% \DeclareDocumentCommand \xi { s } 
% {
%   \IfBooleanTF{#1}
%   {x(i_1,i_2,\dots,i_d)}
%   {x_{i}}
% }



% --- Model Tensor: M ---

% Tensor Model M
\NewDocumentCommand \M {} {\T{M}}

% Mode-k Unfolding of Tensor Model M
\NewDocumentCommand \Mk { O{k} } {\Mx{M}_{(#1)}}
%\NewDocumentCommand \Xk { O{k} } {\Mx{X}_{(#1)}}

% Tensor Model Single Element
\DeclareDocumentCommand \mi { s } 
{
  \IfBooleanTF{#1}
  {m(i_1,i_2\dots,i_d)}
  {m_{i}}
}

% --- Factor Matrix: A_k ---

% Single factor matrix
\NewDocumentCommand \Ak { G{k} t' t"  } { \Mx{A}_{#1}\IfBooleanTF{#2}{^{\intercal}}{}\IfBooleanTF{#3}{^{\phantom{\intercal}}}{} }

% Set of all factor matrices
\NewDocumentCommand \Akset { } {\set{\Ak | k=1,\dots,d}}

% Factor matrix transposed with itself
\NewDocumentCommand \AkAkt { G{k} } {\Ak{#1}'\Ak{#1}"}

% Factor Matrix Element a^(k)_{i_k j}
\NewDocumentCommand \Ake { G{k} G{i} G{j} } {
  a_{#1}(#2_{#1},#3)
}

\NewDocumentCommand \Zk { G{k} t' t"} {\Mx{Z}_{#1}\IfBooleanTF{#2}{^{\intercal}}{}%
  \IfBooleanTF{#3}{^{\phantom{\intercal}}}{}}

% --- Scaling factors: lambda ---

% Entire vector
\NewDocumentCommand \lvec {} {\V{\lambda}}

% Single entry
\NewDocumentCommand \lj { G{j} } {\lambda_{#1}}

% --- Ktensor format ---

% Ktensor (* = weights)
\NewDocumentCommand \KT { s G{d+1}} {
  \llbracket 
  \IfBooleanTF{#1}{\lvec;}{}
  \Ak{1}, \Ak{2}, \dots,  \Ak{#2} \rrbracket
}

\DeclareMathOperator{\nnz}{nnz}

% Commands used from CPRAND
\newcommand{\modeidx}{\ensuremath{m}}
\newcommand{\samplesize}{\ensuremath{S}}
\newcommand{\trans}{\ensuremath{\mathsf{T}}}
\DeclareMathOperator*{\argmin}{arg\,min}
\DeclareMathOperator*{\krp}{KRP}

\NewDocumentCommand \Akj {O{k} G{j}} {\V{a}_{#1}(:,#2)}

\DeclareMathOperator{\rank}{rank}
\DeclareMathOperator{\minimize}{minimize}


\newcommand{\R}{\mathbb{R}}
\newcommand{\E}{\mathbb{E}}
\newcommand{\mcO}{\mathcal{O}}
\newcommand{\Var}{\mathrm{Var}}



% These are the added macros for our notation for the paper
\newcommand{\bperp}{\ensuremath{\V{b}^{\perp}}}
\newcommand{\UA}{\ensuremath{\Mx{U}_{\Mx{A}}}}
\newcommand{\xopt}{\ensuremath{\V{x}_{\text{\rm\sffamily opt}}}}
\newcommand{\xtildeopt}{\ensuremath{\widetilde{\V{x}}_{\text{\rm\sffamily opt}}}}
\newcommand{\ytildeopt}{\ensuremath{\widetilde{\V{y}}_{\text{\rm\sffamily opt}}}}
\newcommand{\yopt}{\ensuremath{\V{y}_{\text{\rm\sffamily opt}}}}

\newenvironment{merged}{\color{gray}}{\color{black}}


%%% Local Variables:
%%% mode: latex
%%% TeX-master: "cprand_sparse"
%%% End:
